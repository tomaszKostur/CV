
%%%%%%%%%%%%%%%%%%%%%%%%%%%%%%%%%%%%%%%%%%
% "ModernCV" CV and Cover Letter
% LaTeX Template
% Version 1.11 (19/6/14)
%
% This template has been downloaded from:
% http://www.LaTeXTemplates.com
%
% Original author:
% Xavier Danaux (xdanaux@gmail.com)
%
% License:
% CC BY-NC-SA 3.0 (http://creativecommons.org/licenses/by-nc-sa/3.0/)
%
% Important note:
% This template requires the moderncv.cls and .sty files to be in the same
% directory as this .tex file. These files provide the resume style and themes
% used for structuring the document.
%
%%%%%%%%%%%%%%%%%%%%%%%%%%%%%%%%%%%%%%%%%

%----------------------------------------------------------------------------------------
%	PACKAGES AND OTHER DOCUMENT CONFIGURATIONS
%----------------------------------------------------------------------------------------

\documentclass[11pt,a4paper,sans]{moderncv} % Font sizes: 10, 11, or 12; paper sizes: a4paper, letterpaper, a5paper, legalpaper, executivepaper or landscape; font families: sans or roman

\moderncvstyle{casual} % CV theme - options include: 'casual' (default), 'classic', 'oldstyle' and 'banking'
\moderncvcolor{blue} % CV color - options include: 'blue' (default), 'orange', 'green', 'red', 'purple', 'grey' and 'black'

\usepackage{lipsum} % Used for inserting dummy 'Lorem ipsum' text into the template

\usepackage{polski}
\usepackage[utf8]{inputenc}
%\usepackage[OT4]{fontenc}

\usepackage[scale=0.75]{geometry} % Reduce document margins
%\setlength{\hintscolumnwidth}{3cm} % Uncomment to change the width of the dates column
%\setlength{\makecvtitlenamewidth}{10cm} % For the 'classic' style, uncomment to adjust the width of the space allocated to your name

%----------------------------------------------------------------------------------------
%	NAME AND CONTACT INFORMATION SECTION
%----------------------------------------------------------------------------------------

\firstname{Tomasz} % Your first name
\familyname{Kostur} % Your last name

% All information in this block is optional, comment out any lines you don't need
\title{Curriculum Vitae}
\address{ul. Nadgoplańska 16B}{88-150 Kruszwica, Polska}
\mobile{+48 669 735 664}
%\phone{}
%\fax{(000) 111 1113}
\email{tomasz.kostur@gmail.com}
\homepage{https://github.com/tomaszKostur}{https://github.com/tomaszKostur} % The first argument is the url for the clickable link, the second argument is the url displayed in the template - this allows special characters to be displayed such as the tilde in this example
%\extrainfo{additional information}
\photo[110pt][0.4pt]{pictures/ja.jpg} % The first bracket is the picture height, the second is the thickness of the frame around the picture (0pt for no frame)
%\quote{"A witty and playful quotation" - John Smith}

%----------------------------------------------------------------------------------------

\begin{document}

\makecvtitle % Print the CV title

%----------------------------------------------------------------------------------------
%	COMPUTER SKILLS SECTION
%----------------------------------------------------------------------------------------

\section{Umiejętności}

\cvitem{\textbf{Pyhton}:} {
    \begin{itemize}
    \item Dobra znajomość ogólna języka.
    \item \textbf{pytest}: tworzenie testów, fixtur, ingerencja w procesy kolekcjonowania przypadków testowych, parametryzacja.
    \item \textbf{pdb/pudb}: swobodne poruszanie się po interfejsie oraz korzystanie ze wszystkich narzędzi debugera.
    \item \textbf{Kivy}: tworzenie aplikacji z dynamicznie dodawanymi lub usuwanymi widgetami, korzystanie ze standardowej pętli zdarzeń.
    \item \textbf{Django}: podstawowa wiedza na temat tworzenia aplikacji oraz założeń frameworku.
    \item \textbf{Sphinx}: ogólna wiedza na temat zasad komentowania kodu kompatybilnego z generatorem dokumentacji, generowanie dokumentacji z użyciem predefiniowanych stylów.
    \end{itemize}
}
\cvitem{\textbf{C++}:} {
    \begin{itemize}
    \item znajomość \textbf{STL}.
    \item podstawowe techniki programowania \textbf{wielowątkowego} przy użyciu standardowej biblioteki C++11.
    \item podstawowa znajomość biblioteki \textbf{OpenCV}.
    \item podstawowa znajomość frameworku \textbf{gtest}.
    \end{itemize}
}
\cvitem{\textbf{C}:} {\begin{itemize}
\item Tworzenie i kompilacja programów na mikrokontrolery AVR z użyciem \textbf{avrlibc}.
\end{itemize}}

\cvitem{Pozostałe:} {
    \begin{itemize}
    \item \textbf{SQL} przy użyciu MariaDB.
    \item \LaTeX.
    \item \textbf{Bash}
    \item Podstawy \textbf{html/css}.
    \item \textbf{Docker} tworzenie oraz używanie pojedynczych kontenerów.
    \item Tworzenie maszyn wirtualnychn za pomocą \textbf{libvirt} we współpracy z \textbf{Qemu}.
    \item Kompilacja za pomocą \textbf{GCC} oraz \textbf{GNU Make}.
    \item Ulubione edytory: \textbf{Vim}, \textbf{Atom}.
    \item \textbf{Golang}, pierwsze kroki w języku.
    \end{itemize}
}

\cvitem{}{Dobra znajomość systemu operacyjnego GNU/Linux.}

%----------------------------------------------------------------------------------------
%	WORK EXPERIENCE SECTION
%----------------------------------------------------------------------------------------

\section{Doświadczenie zawodowe}

\cventry{2015--*}{Software engineer}{Intel Technology Poland}{}{}{Grupy: \begin{itemize}
    \item  \textit{High performance computing} (2015-2017)
    \item \textit{Data center group} (2017-*)
\end{itemize} }
\cvitem{Kompetencje}{
    \begin{itemize}
    \item Tworzenie, testów oprogramowania w językach Pyhton oraz C++.
    \item Monitoring wyników.
    \item Utrzymywanie firmowego frameworku automatycznego wykonywania testów, napisanego w języku Python.
    \item Debugowanie kodu, oraz dostarczanie skryptów reprodukujących błędy.
    \end{itemize}
}
%----------------------------------------------------------------------------------------
%	EDUCATION SECTION
%----------------------------------------------------------------------------------------

\section{Studia i praktyka}

\cventry{2014--2015}{Magisterskie: Automatyka i Robotyka}{Wydział Elektryczny Politechniki Poznańskiej}{\textit{studia dzienne}}{}{Specjalizacja: Robotyka}  % Arguments not required can be left empty
\cventry{2011--2014}{Inżynierskie: Automatyka i Robotyka}{Wydział Elektryczny Politechniki Poznańskiej}{\textit{studia dzienne}}{}{Specjalizacja: Robotyka}

\section{Projekty}

\subsection{Let's Settle}

\cvitem{\textbf{Opis}}{Aplikacja GUI na system Anrdoid, przeliczająca rozliczenia z kosztów wycieczek, pomiędzy użytkownikami. Napisana w języku Python z użyciem biblioteki Kivy.}

\subsection{Projekt programu wizyjnego}

\cvitem{\textbf{Opis}}{Rozpoznawanie numerów tablic rejestracyjnych na podstawie ich zdjęć. Projekt napisany w języku C++ z użyciem biblioteki OpenVC.}

\subsection{Praca Magisterska}

\cvitem{\textbf{Tytuł}}{\emph{Realizacja sprzętowego akceleratora obliczeń działającego pod kontrolą systemu GNU/Linux}}
\cvitem{\textbf{Opis}}{Praca dotyczy realizacji przyspieszenia obliczeń algorytmu stereo-wizyjnego na platformie Xillinx Zynq 7000,
    łączącej w obrębie jednego układu scalonego procesor architektury ARM oraz logikę programowalną.}

\subsection{Praca Inżynierska}

\cvitem{\textbf{Tytuł}}{\emph{Realizacja elektronicznego komutatora modelarskiego silnika BLDC}}
\cvitem{\textbf{Opis}}{Praca dotyczy zagadnienia elektronicznej komutacji bezszczotkotkowego silnika prądu stałego ze sprzężeniem na podstawie sygnałów "back EMF" uzwojeń silnika. Komutator zbudowany został zbudowany na mikrokontrolerze AVR z softem napisanym c języku \textit{C} przy użyciu \textit{avrlibc}.}

%----------------------------------------------------------------------------------------
%	COMMUNICATION SKILLS SECTION
%----------------------------------------------------------------------------------------

\section{Dodatkowe umiejętności}

\cvitem{}{Prawo jazdy kat. A i B}
%\cvitem{2009}{Poster at the Annual Business Conference in Oregon}

%----------------------------------------------------------------------------------------
%	LANGUAGES SECTION
%----------------------------------------------------------------------------------------

\section{Języki}

\cvitemwithcomment{Polski}{język ojczysty}{}
\cvitemwithcomment{Angielski}{poziom średniozaawansowany}{swobodne porozumiewanie się oraz czytanie dokumentacji i artykułów technicznych}
%\cvitemwithcomment{Dutch}{Basic}{Basic words and phrases only}

%----------------------------------------------------------------------------------------
%	INTERESTS SECTION
%----------------------------------------------------------------------------------------

\section{Zainteresowania}

\renewcommand{\listitemsymbol}{-~} % Changes the symbol used for lists

\cvlistdoubleitem{Muzyka filmowa}{Elektronika analogowa}
\cvlistdoubleitem{Wspinaczka skałkowa}{Wspinaczka wysokogórska}

%----------------------------------------------------------------------------------------
%	BONUS ACHIEWMENTS
%----------------------------------------------------------------------------------------
\section{Inne osiągnięcia}
\cvitem{}{Kompozytor, współtwórca muzyki do pełnometrażowego filmu dokumentalnego: \textbf{"GetAway to Pitcairn"}, mającego premierę 15 marca 2015, na festiwalu "Kolosy" w Gdyni.}
\cvitem{}{Kompozytor, współtwórca muzyki do krótkometrażowego filmu dokumentalnego: \textbf{"Bajeczki na dobranoc"}, mającego premierę 2016 roku, na festiwalu "Papaya Films" w Bydgoszczy.}

\footnotesize{
  \vfill
Wyrażam zgodę na przetwarzanie moich danych osobowych zawartych w niniejszym dokumencie dla potrzeb niezbędnych do procesu rekrutacji
(zgodnie z Ustaw a z dnia 29.08.1997 o ochronie danych osobowych, Dziennik Ustaw Nr 133 Poz. 883)
}

\end{document}
\grid
\grid


%%%%%%%%%%%%%%%%%%%%%%%%%%%%%%%%%%%%%%%%%%
% "ModernCV" CV and Cover Letter
% LaTeX Template
% Version 1.11 (19/6/14)
%
% This template has been downloaded from:
% http://www.LaTeXTemplates.com
%
% Original author:
% Xavier Danaux (xdanaux@gmail.com)
%
% License:
% CC BY-NC-SA 3.0 (http://creativecommons.org/licenses/by-nc-sa/3.0/)
%
% Important note:
% This template requires the moderncv.cls and .sty files to be in the same
% directory as this .tex file. These files provide the resume style and themes
% used for structuring the document.
%
%%%%%%%%%%%%%%%%%%%%%%%%%%%%%%%%%%%%%%%%%

%----------------------------------------------------------------------------------------
%	PACKAGES AND OTHER DOCUMENT CONFIGURATIONS
%----------------------------------------------------------------------------------------

\documentclass[11pt,a4paper,sans]{moderncv} % Font sizes: 10, 11, or 12; paper sizes: a4paper, letterpaper, a5paper, legalpaper, executivepaper or landscape; font families: sans or roman

\moderncvstyle{casual} % CV theme - options include: 'casual' (default), 'classic', 'oldstyle' and 'banking'
\moderncvcolor{blue} % CV color - options include: 'blue' (default), 'orange', 'green', 'red', 'purple', 'grey' and 'black'

\usepackage{lipsum} % Used for inserting dummy 'Lorem ipsum' text into the template

\usepackage{polski}
\usepackage[utf8]{inputenc}
%\usepackage[OT4]{fontenc}

\usepackage[scale=0.75]{geometry} % Reduce document margins
%\setlength{\hintscolumnwidth}{3cm} % Uncomment to change the width of the dates column
%\setlength{\makecvtitlenamewidth}{10cm} % For the 'classic' style, uncomment to adjust the width of the space allocated to your name

%----------------------------------------------------------------------------------------
%	NAME AND CONTACT INFORMATION SECTION
%----------------------------------------------------------------------------------------

\firstname{Tomasz} % Your first name
\familyname{Kostur} % Your last name

% All information in this block is optional, comment out any lines you don't need
\title{Curriculum Vitae}
\address{ul. Nadgoplańska 16B}{88-150 Kruszwica, Polska}
\mobile{+48 669 735 664}
%\phone{}
%\fax{(000) 111 1113}
\email{tomasz.kostur@gmail.com}
\homepage{https://github.com/tomaszKostur}{https://github.com/tomaszKostur} % The first argument is the url for the clickable link, the second argument is the url displayed in the template - this allows special characters to be displayed such as the tilde in this example
%\extrainfo{additional information}
\photo[110pt][0.4pt]{pictures/ja.jpg} % The first bracket is the picture height, the second is the thickness of the frame around the picture (0pt for no frame)
%\quote{"A witty and playful quotation" - John Smith}

%----------------------------------------------------------------------------------------

\begin{document}

\makecvtitle % Print the CV title

%----------------------------------------------------------------------------------------
%	WORK EXPERIENCE SECTION
%----------------------------------------------------------------------------------------

\section{Doświadczenie zawodowe}

%\cventry{2015--*}{Software engineer}{Intel Technology Poland}{cztery}{pięć}{}
%----------------------------------------------------------------------------------------
%	COMPUTER SKILLS SECTION
%----------------------------------------------------------------------------------------

\section{Umiejętności}

\cvitem{Dobra znajomość}{ Python, C, C++, Bash}
\cvitem{Podstawowa znajomość}{SQL, golang, html/css, \LaTeX}
\cvitem{Narzędzia, biblioteki języka Python}{pytest, pdb/pudb, Kivy, Django, Sphinx}
\cvitem{Narzędzia, biblioteki języka C lub C++}{OpenCV, avrlibc, STL, gtest}
\cvitem{Narzędzia ogólne}{Docker, ibvirt, Qemu, MariaDB, KiCad, GCC, GNU Make, Vim, Atom}
\cvitem{}{Znajomość systemu operacyjnego GNU/Linux}


%----------------------------------------------------------------------------------------
%	EDUCATION SECTION
%----------------------------------------------------------------------------------------

\section{Studia i praktyka}

\cventry{2014--2015}{Magisterskie: Automatyka i Robotyka}{Wydział Elektryczny Politechniki Poznańskiej}{\textit{studia dzienne}}{}{Specjalizacja: Robotyka}  % Arguments not required can be left empty
\cventry{2011--2014}{Inżynierskie: Automatyka i Robotyka}{Wydział Elektryczny Politechniki Poznańskiej}{\textit{studia dzienne}}{}{Specjalizacja: Robotyka}

\cventry{2013--2015}{Programista PLC}{\textsc{Progterm}}{Konin}{}{}

\cvitem{\textbf{Zakres obowiązków}}{Wykonanie kompletnego programu dla sterownika PLC firmy Carel.
    Program realizuje sterowanie powietrznej pompy ciepła. \newline{}}
\cvitem{\textbf{Cechy programu}}{
\begin{itemize}
\item Komunikacja poprzez protokół MODBUS ze sterownikiem sprężarki pompy ciepła,
\item Konfigurowalny algorytm płynnego zadawania mocy sprężarki,
\item Obsługa i nadzór pracy wejść i wyjść sygnałów analogowych oraz cyfrowych,
\item Sterowanie wewnętrznymi komponentami pompy ciepła takimi jak pompy cyrkulacyjne oraz grzałki,
\item Pełna wizualizacja na Touch-panelu: "Carel PGD Touch".
\end{itemize}
}

%\cventry{2010--2011}{Summer Intern}{}{}{}{}

\section{Projekty}

\subsection{Let's Settle}

\cvitem{\textbf{Opis}}{Aplikacja GUI na system Anrdoid, przeliczjca rozliczenia z kosztow wycieczek, pomiedzy uzytkownikami.}
\cvitem{\textbf{Zakres pracy}}{
    \begin{itemize}
    \item{zaprojektowanie i napisanie programu z jezyku Python,}
    \item{zastosowanie biblioteki Kivy.}
    \end{itemize}}
}

\subsection{Projekt programu wizyjnego}

\cvitem{\textbf{Opis}}{Rozpoznawanie numerów tablic rejestracyjnych na podstawie ich zdjęć.}
\cvitem{\textbf{Zakres pracy}}{
  \begin{itemize}
    \item zaprojektowanie i napisanie programu w języku C++,
    \item zastosowanie biblioteki OpenCV,
    \item zastosowanie autorskiego modułu OCR.
    \end{itemize}}

\subsection{Praca Magisterska}

\cvitem{\textbf{Tytuł}}{\emph{Realizacja sprzętowego akceleratora obliczeń działającego pod kontrolą systemu GNU/Linux}}
\cvitem{\textbf{Opis}}{Praca dotyczy realizacji przyspieszenia obliczeń algorytmu stereowizyjnego na platformie Xillinx Zynq 7000,
    łączącej w obrębie jednego układu scalonego procesor architektury ARM oraz logikę programowalną.}
\cvitem{\textbf{Zakres pracy}}{
    \begin{itemize}
    \item kompilacja systemu opartego na jądrze Linux na platformę w architekturze ARM,
    \item zaprojektowanie sprzętowego akceleratora w języku Verilog,
    \item dostarczenie napisanego w C API dla zaprojektowanego akceleratora,
    \item porównanie otrzymanych wyników wydajnościowych z analogicznymi funkcjonalnościami zawartymi w bibliotece OpenCV.
    \end{itemize}}



\subsection{Praca Inżynierska}

\cvitem{\textbf{Tytuł}}{\emph{Realizacja elektronicznego komutatora modelarskiego silnika BLDC}}
\cvitem{\textbf{Opis}}{Praca dotyczy zagadnienia elektronicznej komutacji bezszczotkotkowego silnika prądu stałego ze sprzężeniem na podstawie sygnałów "back EMF" uzwojeń silnika.}
\cvitem{\textbf{Zakres pracy}}{
  \begin{itemize}
    \item zaprojektowanie analogowo-cyfrowego układu elektronicznego realizującego zadanie sterowania,
    \item zaprojektowanie obwodu elektronicznego w technologii PCB,
    \item napisanie programu na mikrokontroler z rodziny ATmega w języku C.
    \end{itemize}}

%----------------------------------------------------------------------------------------
%	COMMUNICATION SKILLS SECTION
%----------------------------------------------------------------------------------------

\section{Dodatkowe umiejętności}

\cvitem{}{Prawo jazdy kat. A i B}
%\cvitem{2009}{Poster at the Annual Business Conference in Oregon}

%----------------------------------------------------------------------------------------
%	LANGUAGES SECTION
%----------------------------------------------------------------------------------------

\section{Języki}

\cvitemwithcomment{Polski}{język ojczysty}{}
\cvitemwithcomment{Angielski}{poziom średniozaawansowany}{swobodne porozumiewanie się oraz czytanie dokumentacji i artykułów technicznych}
%\cvitemwithcomment{Dutch}{Basic}{Basic words and phrases only}

%----------------------------------------------------------------------------------------
%	INTERESTS SECTION
%----------------------------------------------------------------------------------------

\section{Zainteresowania}

\renewcommand{\listitemsymbol}{-~} % Changes the symbol used for lists

\cvlistdoubleitem{Muzyka filmowa}{Elektronika analogowa}
\cvlistitem{Wspinaczka skałkowa}{Wspinaczka wyskogórska}

\section{Inne osiągnięcia}
\cvitem{}{Kompozytor muzyki do pełnometrażowego filmu dokumentalnego: "GetAway to Pitcairn" mającego premierę 15 marca 2015, na festiwalu "Kolosy" w Gdyni.}

\footnotesize{
  \vfill
Wyrażam zgodę na przetwarzanie moich danych osobowych zawartych w niniejszym dokumencie dla potrzeb niezbędnych do procesu rekrutacji
(zgodnie z Ustaw a z dnia 29.08.1997 o ochronie danych osobowych, Dziennik Ustaw Nr 133 Poz. 883)
}
%----------------------------------------------------------------------------------------
%	COVER LETTER
%----------------------------------------------------------------------------------------

% To remove the cover letter, comment out this entire block

%\clearpage
%
%\recipient{HR Department}{Corporation\\123 Pleasant Lane\\12345 City, State} % Letter recipient
%\date{\today} % Letter date
%\opening{Dear Sir or Madam,} % Opening greeting
%\closing{Sincerely yours,} % Closing phrase
%\enclosure[Attached]{curriculum vit\ae{}} % List of enclosed documents
%
%\makelettertitle % Print letter title
%
%\lipsum[1-3] % Dummy text
%
%\makeletterclosing % Print letter signature

%----------------------------------------------------------------------------------------

\end{document}
\grid
\grid
